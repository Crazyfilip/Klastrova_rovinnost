\documentclass[12pt,a4report]{report}
\usepackage[T1]{fontenc}
\usepackage[czech]{babel}
\usepackage[utf8]{inputenc}
\usepackage{amsthm}
\usepackage{amsmath}
\usepackage{tikz}
\usepackage{float}

\usepackage{lineno}\linenumbers

\newtheorem{theorem}{Věta}[chapter]
\newtheorem{tvr}[theorem]{Tvrzení}
\newtheorem{lemma}[theorem]{Lemma}
\newtheorem{dusl}[theorem]{Důsledek}

\theoremstyle{definition}
\newtheorem{defn}[theorem]{Definice}

\begin{document}
\author{Filip Šedivý}

\chapter{Minorové operace}

V této kapitole zavedeme operace s klastrovým grafem, které zachovávají klastrovou rovinnost. V závěru kapitoly zavedeme pojem klastrového minoru jakožto hlavní pojem této kapitoly, jenž v následující kapitole použijeme pro charakterizaci zakázaných minorů omezených problémů klastrové rovinnosti.

\begin{defn} (minorové operace) Mějme klastrový graf $(G, \mathcal C)$. Minorové operace na klastrových grafech jsou následující:

(Pozn.: Pokud se neřekne jinak, operace se smí provést v nakreslené i nenakreslené verzi)

\textit{Odebráním vrcholu $v$} z klastrového grafu $(G, \mathcal C)$ vznikne klastrový graf  $(G', \mathcal C')$, kde  $G' = (V \setminus \{v\}, E \setminus \{f |$ hrana $f$ obsahovala vrchol $v\})$ a $\mathcal C'$ je klastrová hierarchie, kde se z klastrů odebere vrchol $v$, pokud v nich byl.

\textit{Odebráním hrany $e$} z klastrového grafu $(G, \mathcal C)$ vznikne klastrový graf  $(G', \mathcal C)$, kde $G' =  (V,E \setminus \{e\})$.

\textit{Odebráním klastru $K$} z klastrového grafu $(G, \mathcal C)$ vznikne klastrový graf  $(G, \mathcal C')$, kde $\mathcal C' = \mathcal C \setminus K$.

\textit{Kontrakcí hrany $e=\{x,y\}$} z klastrového grafu $(G, \mathcal C)$ vznikne klastrový graf  $(G', \mathcal C')$, kde $G'$ je graf, který obdržíme kontrakcí hrany $e$ a $\mathcal C'$ získáme nahrazením vrcholů $x$ a $y$ ve všech klastrech, kde byli, vrcholem vzniklým kontrakcí. Kontrakci můžeme provést za předpokladu, že koncové vrcholy $x, y$ leží ve stejných klastrech.

\textit{Nahrazením klastru $K=\{x,y\}$ o velikosti 2 hranou $e$} z klastrového grafu klastrového grafu $(G, \mathcal C)$ vznikne klastrový graf  $(G', \mathcal C')$, kde $G'=(V,E \cup e)$ a $\mathcal C'= C \setminus K$. Předpokládáme, že vrcholy $x, y$ nejsou spojeny hranou, neboť v tom případě tato operace není potřebná. U nakreslené verze navíc předpokládáme, že $e$ má jednoznačně dané kombinatorické nakreslení.

\textit{Odebráním vrcholu $v$ z klastru $K$} z klastrového grafu $(G, \mathcal C)$ vznikne klastrový graf  $(G, \mathcal C')$, kde $\mathcal C'$ vznikne nahrazením klastru $K$ klastrem $K \setminus \{v\}$. Tuto operaci lze provést za předpokladu, že z vrcholu $v$ vychází právě jedna hrana ven z $K$ a $K$ je nejmenší (vzhledem k inkluzi) klastr obsahující $v$.

\textit{Sjednocení disjuktních klastrů $K_1$ a $K_2$} z klastrového grafu $(G, \mathcal C)$ vznikne klastrový graf  $(G, \mathcal C')$, kde $C'  := (C\setminus \{K_1,K_2\}) \cup \{K_1 \cup K_2\}$. Sjednocení klastrů můžeme provést za předpokladů, že $K_1, K_2$ jsou dva minimální klastry (nemají podklastry) se společným rodičem, $K_1 \cup K_2$ neindukuje kružnici s vrcholem mimo $K_1 \cup K_2$ uvnitř. Jinými slovy $K_1 \cup K_2$ nemá díru v $G$ (v nakreslené verzi). V nenakreslené verzi je podmínkou, že existuje nakreslení $\rho$ grafu $G$ takové, že $K_1 \cup K_2$ nemá díru v $\rho$. Posledním předpokladem je, že existuje hrana spojující $K_1$ s $K_2$.
\end{defn}

Název minorové operace je užit proto, že každá operace zjednoduše daný vstupní klastrový graf. U sjednocení klastrů v nenakresleném klastrovém grafu je obtížné říci, kdy potřebné nakreslení existuje, to činí operaci méně použitelnou.
Nyní vyzkoumáme dopad minorových operací na klastrový graf, konkrétně dopad na existenci klastrového nakreslení.

\begin{tvr} Nechť $(G', \mathcal C')$ vznilkne z $(G, \mathcal C)$ minorovou operací. Potom pokud $(G, \mathcal C)$ je klastrově rovinný, tak $(G', \mathcal C')$ je klastrově rovinný.
\label{min_op_zach_kl_rov}
\end{tvr}
\begin{proof}

Mějme klastrový graf $(G, \mathcal C)$ a jeho klastrové nakreslení $\rho$.

Odebrání vrcholu, hrany, klastru zachovává klastrovou rovinnost:\\
Mějme dáno klastrové nakreslení. Odebrání hrany zapříčiní jedině to, že se nemusí v daném nakreslení hrana kreslit. Podobně pro odebraný vrchol, kdy se odebereu hrany vedoucí do něj. Odebraný klastr se též prostě nenakreslí.

Kontrakce hrany zachovává klastrovou rovinnost:\\
Mějme dáno klastrové nakreslení. Kontrakce je jen vlastně smrštění hrany do jediného bodu, jenž zastupuje vrchol vzniklý kontrakcí.

Nahrazení klastru $K=\{x,y\}$ hranou $e$ zachovává klastrovou rovinnost:\\
Jelikož $(G, \mathcal C)$ je klastrově rovinný, tak máme saturátor $S$. Jelikož $K$ je podle předpokladu nesouvislý, tak po přidání saturátoru jsou vrcholy $x, y$ spojeny hranou. Tato hrana ze saturátoru spojující $x,y$ je hledanou hranou $e$. V nakreslené verzi je požadavek na jednoznačnost, protože by se jinak mohlo stát, že nahrazením klastru hranou vznikne díra.  

Odebraní vrcholu z klastru zachovává klastrovou rovinnost:\\
Jednoduchý překreslovací argument, kdy podél hrany protáhneme hranici klastru až ji přetáhneme přes vyjímaný vrchol.

Sjednocení klastrů zachovává klastrovou rovinnost:\\
Nechť $S$ je minimální saturátor $(G,\mathcal C)$ takový, že $(G \cup S,\mathcal C)$ nemá díru. $S$ je saturátorem i pro klastrový graf $(G, \mathcal C')$, kde ale může být díra.
Nechť $S'$ je  minimální saturátor $(G, \mathcal C')$ a $S' \subseteq S$. Tvrdíme, že $(G \cup S',\mathcal C')$ nemá díru. To dokážeme sporem.

Nechť $D$ je díra. Ta musí být ve sjednocení klastrů $K_1$ a $K_2$, neboť kdyby byla jinde, bylo by to ve sporu s předpokladem, že $(G \cup S, \mathcal C)$ nemá díru.
Díra $D$ má neprázdný průnik se saturátorem $S'$. Kdyby průnik byl prázdný, znamenalo by to, že příslušná díra byla v původním klastrovém grafu. Označme tuto hranu $e = \{x,y\}$, kde $x$ a $y$ jsou její koncové vrcholy.
Jako $S''$ označme $S' \setminus e$. Množina $S''$ je saturátorem, protože každý klastr $K \in C$ obsahující vrcholy  $x$ a $y$ obsahuje i cestu $D \setminus \{e\}$. Dostali jsme tedy spor s minimalitou $S'$. $S'$ tedy neobsahuje díry.
\end{proof}

Sjednocení klastrů má dost přepodkladů, a proto uvedeme, proč jsou tyto předpoklady nutné. Předpoklad o společném rodiči je z důvodu zachování klastrové hierarchie. Následující příklad klastrového grafu (obrázek \ref{podklastr}) ukazuje, proč je nutný předpoklad o tom, že sjednocované klastry nesmějí mít podklastry.

\begin{figure}[H]
\begin{tikzpicture}[node/.style={circle,fill=black!20,draw,minimum size=1em,inner sep=3pt]}]
  
  \node[node] (1) at  (0,0) {};
  \node[node] (2) at  (0,-2) {};
  \node[node] (3) at  (1.3,-1.2) {};
  \node[node] (4) at  (2.3,0.1) {};
  \node[node] (5) at  (2.6,-0.9) {};
  \node[node] (6) at  (2.7,-2) {};
  \node[node] (7) at  (4,1.35) {};

  \draw (1) -- (2);
  \draw (1) -- (3);
  \draw (1) -- (4);
  \draw (1) -- (7);
  \draw (2) -- (3);
  \draw (2) -- (6);
  \draw (3) -- (4);
  \draw (3) -- (6);
  \draw (4) -- (7);
  \draw (5) -- (6);
  \draw (5) -- (7);
  \draw (6) -- (7);

  \draw[dashed] (-0.4,0.4) to node [auto,swap] {$K_1$} (-0.4,-2.4);
  \draw[dashed] (-0.4,0.4) -- (0.4,0.4);
  \draw[dashed] (0.4,0.4) -- (0.4,-2.4);
  \draw[dashed] (-0.4,-2.4) -- (0.4,-2.4);

  \draw[dashed] (1.9,-1.2) -- (1.9,0.4);
  \draw[dashed] (1.9,0.4) -- (2.9,0.4);
  \draw[dashed] (2.9,0.4)  to node [auto,swap] {$K_3$} (2.9,-1.2);
  \draw[dashed] (2.9,-1.2) -- (1.9,-1.2);

  \draw[dashed] (1.75,-2.3) to node [auto,swap] {$K_2$} (3.1,-2.3);
  \draw[dashed] (3.1,-2.3) -- (3.1,0.52);
  \draw[dashed] (3.1,0.52) --  (1.75,0.52);
  \draw[dashed]  (1.75,0.52) -- (1.75,-2.3);
\end{tikzpicture}
\caption{Klastr $K_3$ brání sjednocení klastrů $K_1$ a $K_2$, neboť jeho saturováním (nahrazení hranou) by v $K_1 \cup K_2$ vznikla díra. Kdybychom vynechali $K_3$, pak už je snadné najít klastrové nakreslení.}
\label{podklastr}
\end{figure}

Nyní se podíváme na dva speciální případy sjednocení klastrů. Jeden případ je přidání vrcholu do klastru, který je vlastně inverzí k odebrání vrcholu z klastru. 

\begin{tvr} Mějme klastrový graf $(G, \mathcal C)$ a vrchol $v \in V(G)$ a klastr $K \in \mathcal C$. Nechť $K$ neobsahuje podklastry a každý klastr obsahující $v$ obsahuje i $K$, $v$ sousedí s $K$, tedy $v$ je spojen s nějakým vrcholem v $K$ hranou a
$K \cup \{v\}$ neindukuje kružnici s vrcholem mimo $K$ uvnitř.
$\mathcal C' = (\mathcal C \setminus \{K\}) \cup \{ K \cup \{v\} \}$
Potom $(G, \mathcal C)$ je klastrově rovinný $\implies (G, \mathcal C')$ je klastrově rovinný.
\end{tvr}
\begin{proof}
Jednoduše, budeme vrchol vydávat za jednovrcholový klastr. Zbytek plyne z toho, že sjednocení zachovává klastrovou rovinnost, jelikož jsou splněny všechny předpoklady.
\end{proof}

Podobně pro nenakreslenou verzi. Druhý případ je, když klastry spojuje právě jedna hrana. Tento případ nám dává příklad, kdy můžeme provést sjednocení klastrů i pro nenakreslené klastrové grafy.
\begin{tvr}
Mějme klastrový graf  $(G, \mathcal C)$ a dva disjunktní klastry $K_1$ a $K_2$, které spojuje právě jedna hrana $e$. Nechť $K_1$ a $K_2$ mají společného rodiče a nemají podklastry. Mějme klastrový graf $(G, \mathcal C')$, kde  $\mathcal C'  := (\mathcal C \setminus \{K_1,K_2\}) \cup \{K_1 \cup K_2\}$. Potom $(G, \mathcal C)$ je klastrově rovinný $\implies (G, \mathcal C')$ je klastrově rovinný.
\end{tvr}
\begin{proof}
Pro použití výsledku o sjednocení klastrů nám stačí ukázat, že $K_1 \cup K_2$ neobsahuje díru. Protože klastry $K_1$ a $K_2$ spojuje právě jedna hrana, tak jediné kružnice ve sjednocení jsou buď $K_1$ nebo v $K_2$. Podle předpokladu, že $(G, \mathcal C)$ je klastrově rovinný, tak $K_1$ ani $K_2$ neobsahují díru. Podle tvrzení \ref{min_op_zach_kl_rov} je $\implies (G, \mathcal C')$ klastrově rovinný.
\end{proof}

Vyzbrojeni minorovými operacemi můžeme definovat pojem klastrového minoru

\begin{defn}
Mějme klastrový graf $(G,\mathcal C)$. Klastrový graf $(G',\mathcal C')$ je klastrovým minorem, pokud jej lze získat konečnou posloupností minorových operací z klastrového grafu $(G,\mathcal C)$.
\end{defn}

\begin{dusl} Klastrový minor klastrově rovinného klastrového grafu je klastrově rovinný.
\label{důsledek}
\end{dusl}
\begin{proof}
Důkaz se provede indukcí podle délky posloupnosti, kde se využije toho, že minorové operace zachovávají klastrovou rovinnost.
\end{proof}
\end{document}