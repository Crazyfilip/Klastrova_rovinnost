\documentclass[12pt,a4report]{report}
\usepackage[czech]{babel}
\usepackage[utf8]{inputenc}
\usepackage{amsthm}
\usepackage{amsmath}


\newtheorem{def1}{Definice (minorové operace)}
\newtheorem{poz1}{Poznámka k definici 6. minorové operace}
\newtheorem{tvr1}{Tvrzení}
\newtheorem{lemma1}{Lemma}

\title {Minorové operace}

\begin{document}
\author{Filip Šedivý}
\maketitle

\chapter{Základní definice}

Pokusný text

\begin{def1}
Minorové operace jsou následující:
\begin{enumerate}
\item Odebrání hrany nebo vrcholu z grafu
\item Odebrání klastru z klastrové hierarchie
\item Kontrakce hrany
\begin{itemize}
\item  pokud oba konce hrany patří do stejných klastrů
\end{itemize}
\item Nahrazení klastru velikosti 2 hranou
\begin{itemize}
\item  v nakreslené verzi problému jenom tehdy, pokud to lze provést jednoznačně
\end{itemize}
\item Odebrání vrcholu v z klastru K
\begin{itemize}
\item  K je nejmenší (vzhledem na inkluzi) klastr obsahující v
\item z v vychází právě 1 hrana ven z K 
\end{itemize}
\item Sjednocení dvou disjunktní klastrů
\begin{itemize}
\item  Spojuje je aspoň 1 hrana
\item Klastry nejsou podklastry jiných klastrů
\end{itemize}
\end{enumerate}
\end{def1}

\begin{poz1}
S jistotou funguje, pokud je spojuje právě 1 hrana. Pro více hran to je problematické.
\end{poz1}

\begin{tvr1}
Minorové operace (1-5) zachovávají klastrovou rovinnost
\end{tvr1}
\begin{proof}
\end{proof}

\begin{lemma1} Připojení vrcholu do klastru \\
(G,C) "nakreslená" instance klastrové rovinnosti, v $\in$ V(G) a K $\in$ C \\
C' = $(C \verb=\= \{K\}) \cup \{ K \cup \{v\} \}$
\begin{enumerate}
\item v sousedí s K (je spojen s nějakým vrcholem v K hranou)
\item Každý klastr obsahující v obsahuje i K
\item K nemá podklastry
\item K $\cup \{v\}$ neindukuje kružnici s vrcholem mimo K uvnitř
\end{enumerate}
Potom (G,C) je kl. rovinný $\implies$ (G,C') je kl. rovinný
\end{lemma1}
\begin{proof}
\end{proof}

\end{document}