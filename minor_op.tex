\documentclass[12pt,a4report]{report}
\usepackage[T1]{fontenc}
\usepackage[czech]{babel}
\usepackage[utf8]{inputenc}
\usepackage{amsthm}
\usepackage{amsmath}

\usepackage{lineno}\linenumbers

\newtheorem{defn}{Definice: }[chapter]
\newtheorem{tvr}{Tvrzení}[chapter]
\newtheorem{lemma}{Lemma}[chapter]

\begin{document}
\author{Filip Šedivý}

\chapter{Minorové operace}

V této kapitole zavedeme operace s klastrovým grafem, které zachovávají klastrovou rovinnost. V závěru kapitolu zavedeme pojem klastrové minoru jakožto

\begin{defn} (minorové operace)
Minorové operace jsou následující:
\begin{enumerate}
\item Odebrání hrany nebo vrcholu z grafu
\item Odebrání klastru z klastrové hierarchie
\item Kontrakce hrany
\begin{itemize}
\item  pokud oba konce hrany patří do stejných klastrů
\end{itemize}
\item Nahrazení klastru velikosti 2 hranou
\begin{itemize}
\item  v nakreslené verzi problému jenom tehdy, pokud to lze provést jednoznačně
\end{itemize}
\item Odebrání vrcholu v z klastru K (v nakreslené verzi)
\begin{itemize}
\item  K je nejmenší (vzhledem na inkluzi) klastr obsahující v
\item z v vychází právě 1 hrana ven z K 
\end{itemize}
\item Sjednocení dvou disjunktní klastrů $K_1, K_2 \in \mathcal C$  (nakreslená verze)
\begin{itemize}
\item $K_1, K_2$ jsou dva minimální klastry (nemají podklastry) se společným rodičem
\item $K_1 \cup K_2$ neindukuje kružnici s vrcholem mimo $K_1 \cup K_2$ uvnitř. Jinými slovy $K_1 \cup K_2$ nemá díru v G.
\item existuje hrana spojující $K_1$ s $K_2$
\item $C'  := (C\setminus \{K_1,K_2\}) \cup \{K_1 \cup K_2\}$
\end{itemize}
\end{enumerate}
\end{defn}

\begin{tvr}
Minorové operace (1) a (2) zachovávají klastrovou rovinnost
\end{tvr}
\begin{proof}
Mějme dáno klastrové nakreslení. Odebrání hrany zapříčiní jedině to, že se nemusí v daném nakreslení hrana kreslit. Podobně pro odebraný vrchol, kdy se odebereu hrany vedoucí do něj. Odebráný klastr se též prostě nenakreslí
\end{proof}

\begin{tvr}
Kontrakce hrany (3) zachovává klastrovou rovinnost
\end{tvr}
\begin{proof}
Mějme dáno klastrové nakreslení. Kontrakce je jen vlastně smrštění hrany do jediného bodu, jenž zastupuje vrchol vzniklý kontrakcí.
\end{proof}

\begin{tvr}
Operace (4) zachovává klastrovou rovinnost
\end{tvr}
\begin{proof}
Stačí si uvědomit, že takový klastr se chová jako hrana. V nakreslené verzi je požadavek na jednoznačnost (jen jediná stěna, kde lze hranu dokreslit), protože by jinak se mohlo stát nahrazením klastru hranou, že vznikne díra.  
\end{proof}

\begin{tvr}
Odebraní vrcholu z klastru (5) zachovává klastrovou rovinnost.
\end{tvr}
\begin{proof}
Jednoduchý překreslovací argument, kdy podél hrany protáhneme hranici klastru až ji přetáhneme přes vyjímaný vrchol. (TODO dát ilustrativní obrázek)
\end{proof}

\begin{tvr} Připojení vrcholu do klastru \\
(G,C) "nakreslená" instance klastrové rovinnosti, v $\in$ V(G) a K $\in$ C \\
C' = $(C \verb=\= \{K\}) \cup \{ K \cup \{v\} \}$
\begin{enumerate}
\item v sousedí s K (je spojen s nějakým vrcholem v K hranou)
\item Každý klastr obsahující v obsahuje i K
\item K nemá podklastry
\item K $\cup \{v\}$ neindukuje kružnici s vrcholem mimo K uvnitř
\end{enumerate}
Potom (G,C) je kl. rovinný $\implies$ (G,C') je kl. rovinný
\end{tvr}
\begin{proof}
Toto tvrzení je speciálním případem následujícího lemmatu. Jednoduše, budeme vrchol vydávat za jednovrcholový klastr.
\end{proof}

\begin{tvr} Sjednocení klastrů (6) zachovává klastrovou rovinnost.
\end{tvr}
\begin{proof}
Nechť S je minimální saturátor $(G,\mathcal C)$ takový, že $(G[V,E \cup S],\mathcal C)$ nemá díru. S je saturátorem i pro klastrový graf $(G, \mathcal C')$, kde ale může být díra.
Nechť existuje minimální saturátor $S' \subseteq S$ takový, že  $(G[V,E \cup S'],\mathcal C)$ nemá díru. To dokážeme sporem.

Nechť D je díra. Ta musí být ve sjednocení klastrů $K_1$ a $K_2$, neboť kdyby byla jinde, bylo by to ve sporu s předpokladem, že původní klastrový graf je klastrově rovinný.
Díra D má neprázdný průnik se saturátorem S'. Kdyby průnik byl prázdný, znamenalo by to, že příslušná díra byla v původním klastrovém grafu. Označme tuto hranu $e = \{x,y\}$, kde $x$ a $y$ jsou její koncové vrcholy.
Jako S'' označme $S' \setminus e$. Množina S'' je saturátorem, protože každý klastr $K \in C'$ obsahující vrcholy  $x$ a $y$ obsahuje i cestu $D \setminus \{e\}$. Dostali jsme tedy spor s minimalitou S'. S' tedy neobsahuje díry.
\end{proof}

(TODO vysvětlení předpokladů)

Vyzbrojeni minorový operace můžeme definovat pojem klastrového minoru

\begin{defn}
Mějme klastrový graf $(G,\mathcal C)$. Klastrový graf $(G',\mathcal C')$ je klastrovým minorem, pokud jej lze získat konečnou posloupností minorových operací z klastrového grafu $(G,\mathcal C)$.
\end{defn}
\end{document}