Práce se zabývá problémem klastrové rovinnosti a ubírá se dvěma směry. Jedním směrem je otázka výpočetní složitosti, kde ukážeme, jak klastrovou rovinnost řešit v lineárním nedeterministickém čase vzhledem k počtu vrcholů vstupního grafu. Druhým směrem je charakterizace omezených verzí klastrové rovinnosti pomocí minimálních klastrově nerovinných instancí. Za tím účelem definujeme pojem klastrového minoru pomocí několika operací redukujících klastrové grafy, o nichž dokážeme, že zachovávají klastrovou rovinnost. Dokážeme, že v případě klastrového grafu, kde klastry mají velikost 2 a grafem je kružnice nebo cesta, existuje jen konečně mnoho minimálních klastrově nerovinných minorů. Kromě toho uvedeme i již známe výsledky o výpočetní složitosti klastrové rovinnosti.