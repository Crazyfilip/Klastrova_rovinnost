\documentclass[12pt,a4report]{report}
\usepackage[czech]{babel}
\usepackage[utf8]{inputenc}
\usepackage{amsthm}
\usepackage{amsmath}


\newtheorem{tvr1}{Tvrzení}
\newtheorem{tvr2}{Tvrzení}
\newtheorem{tvr3}{Tvrzení}
\newtheorem{tvr4}{Tvrzení}

\begin{document}
\author{Filip Šedivý}


\chapter{Složitost}
\par
V této kapitole ukážeme několik výsledků ohledně časové a prostorové složitosti.
Problém klastrové rovinnosti patří do třídy NP z pohledu časové složitosti a z hlediska prostového se dá řešit v prostoru $\mathcal{O}(n)$ na deterministiském stroji. Jako výchozí model uvažujeme RAM, případně jeho nedeterministickou verzi nRAM.

\section{Datová reprezentace}
Nejprve uvedeme možnosti reprezentace klastrového grafu a ujasníme vzhledem k čemu budeme vztahovat příslušnou složitost. 
Velikost grafu na vstupu je  $\mathcal{O}(n+m+\|C\|)$, kde C je klastrová hierarchie. 
Jelikož graf musí být rovinný, platí pro m, že $m=\mathcal{O}(n)$.
Nejprve musíme určit kolik klastrů se v klastrové hierarchii může nacházet.
Pro zjednodušení budeme předpokládat, že klastr může obsahovat jediný vrchol nebo i všechny.
\begin{tvr1}
Maximální počet klastrů v grafu G s n (n $\geq$ 1) vrcholy je 2n-1.
\end{tvr1}
\begin{proof}
Důkaz indukcí podle n: \\
Základ indukce : n=1 \\
Zjevně platí. \\
Indukční předpoklad: Tvrzení platí pro $|V| <$ n.\\
Indukční krok:  \\
BÚNO: každý vrchol je minimálně obsažen v klastru obsahující pouze jej.
Díky tomuto pak každý klastr, který obsahuje aspoň dva vrcholy se rozkládá aspoň na dva menší obsahující méně vrcholů. \\
Máme graf s n vrcholy. Podle předpokladu máme klastr K obsahující všechny vrcholy. Ten obsahuje k vzájemně disjunktních podklastrů (takových, že už jediný klastr, ve kterém jsou obsaženy je K). Velikost i-tého klastru nechť je k$_{\text{i}}$ Každý z těchto klastrů obsahuje méně než n vrcholů. Platí pro ně tedy indukční předpoklad. Máme tedy: \\
$\#$ max. počet klastrů = 1 + $\sum\limits_{i=1}^k (2*k_{\text{i}}-1) = 1 + 2*\sum\limits_{i=1}^k k_{\text{i}} - k = 2n - k + 1 $\\
K maximalizování dojde pokud bude vždy k=2.
\end{proof}

Pro reprezentaci klastrové hierarchie se nabízí dvě možnosti.
\begin{enumerate}
\item Seznamy vrcholů
\item Strom, kde listy představují vrcholy a vnitřní uzly představují klastry
\begin{itemize}
\item Zde předpokládejme, že kořen tohoto stromu reprezentuje klastr obsahující všechny vrcholy.
\end{itemize}
\end{enumerate}

První varianta má za následek, že klastrová hierarchie zabírá prostor až $\mathcal{O}(n^2)$. Příkladem takové klastrové hierarchie je graf, kde klastry jsou postupně do sebe vnořené. První klastr obsahuje všechny vrcholy, druhý o vrchol méně, třetí o další vrchol, ... .
Druhá varianta naproti tomu dává prostor $\mathcal{O}(n)$. Stačí tedy určovat složitost (časovou a paměťovou) vzhledem k počtu vrcholů vstupního grafu.

\section{časová složitost}
Hlavním výsledkem této části je lineární nedeterministický algoritmus pro klastrovou rovinnost.
\begin{tvr2}
Problém rozhodnotí existence rovinného klastrového nakreslení patří do třídy NP.
\end{tvr2}
\begin{proof}
Využíváme toho, že ekvivalentním problémem ke klastrové rovinnosti, je existence saturátoru. Ten nám zajistí, že klastry jsou souvislé. Saturátor dostaneme jako certifikát. Vzhledem k tomu, že klastrů je polynomiálně mnoho, tak ověření saturátoru se dá provést  v polynomiálním čase (Například otestování souvislosti každého klastru zvláště). Dále jsou algoritmy testují klastrovou rovinnost v polynomiálním čase (TODO doplnit reference), pokud klastry jsou souvislé. 
\end{proof}

\begin{tvr3}
Pro problém klastrové rovinnosti je nedeterministický algoritmus, jehož časová složitost je $\mathcal{O}(n)$.
\end{tvr3}
\begin{proof}
Důkaz tohoto tvrzení je pouze doplněním důkazu, že klastrová rovinnost je v NP. Pro důkaz je třeba ukázat, že umíme ověřit souvislost všech klastrů v čase $\mathcal{O}(n)$, a pak že klastrová rovinnost se dá otestovat v lineárním čase pokud jsou klastry souvislé.
Druhá část viz (TODO doplnit reference)
(TODO dodělat důkaz, souvislost klastrů)
\end{proof}

\section{prostorová složitost}
Z výsledků o časové složitosti můžeme říci, že můžeme klatrovou rovinnost rozhodovat v nedeterministickém prostoru o velikosti $\mathcal{O}(n)$. Ze Savitchovy věty (doplnit ref) plyne, že v deterministickém prostoru stačí nejvýše prostor velikosti $\mathcal{O}(n^2)$.
Lepšího výsledku, ve smyslu, že potřebujeme méně prostoru, dosáhneme využítím vztahu tříd NTIME a DSPACE, který je $NTIME(t(n)) \subseteq DSPACE(t(n))$. Jelikož máme nedeterministický algoritmus pro klastrovou rovinnost pracující v lineárním čase, tak díky předešlému víme, že na existuje deterministický algoritmus využívající pouze lineárně mnoho prostoru.
\begin{tvr4}
Klastrová rovinnost lze rozhodovat na RAMu s lineárně omezeným prostorem.
\end{tvr4}
\begin{proof}
\end{proof}
\end{document}