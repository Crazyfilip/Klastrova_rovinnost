\documentclass[12pt,a4report]{report}
\usepackage[czech]{babel}
\usepackage[utf8]{inputenc}
\usepackage{amsthm}
\usepackage{amsmath}


\newtheorem{tvr1}{Tvrzení}

\title {Složitost}

\begin{document}
\author{Filip Šedivý}
\maketitle

Pro účely následujucího tvrzení předpokládejme, že klastr může obsahovat jediný vrchol nebo i všechny.
\begin{tvr1}
Maximální počet klastrů v grafu G s n (n $\geq$ 1) vrcholy je 2n-1.
\end{tvr1}
\begin{proof}
Důkaz indukcí podle n: \\
Základ indukce : n=1 \\
Zjevně platí. \\
Indukční předpoklad: Tvrzení platí pro $|V| <$ n.\\
Indukční krok:  \\
BÚNO: každý vrchol je minimálně obsažen v klastru obsahující pouze jej.
Díky tomuto pak každý klastr, který obsahuje aspoň dva vrcholy se rozkládá aspoň na dva menší obsahující méně vrcholů. \\
Máme graf s n vrcholy. Podle předpokladu máme klastr K obsahující všechny vrcholy. Ten obsahuje k vzájemně disjunktních podklastrů (takových, že už jediný klastr, ve kterém jsou obsaženy je K). Velikost i-tého klastru nechť je k$_{\text{i}}$ Každý z těchto klastrů obsahuje méně než n vrcholů. Platí pro ně tedy indukční předpoklad. Máme tedy: \\
$\#$ max. počet klastrů = 1 + $\sum\limits_{i=1}^k (2*k_{\text{i}}-1) = 1 + 2*\sum\limits_{i=1}^k k_{\text{i}} - k = 2n - k + 1 $\\
K maximalizování dojde pokud bude vždy k=2.
\end{proof}

\end{document}