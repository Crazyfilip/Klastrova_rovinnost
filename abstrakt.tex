\documentclass[12pt,a4report]{report}
\usepackage[czech]{babel}
\usepackage[T1]{fontenc}
\usepackage[utf8]{inputenc}

\begin{document}
\author{Filip Šedivý}

Práce se zabývá problémem klastrové rovinnosti a ubírá se dvěma směry. Jedním směrem je otázka výpočetní složitosti, kde ukážeme, jak klastrovou rovinnost řešit v lineárním nedeterministickém čase vzhledem k počtu vrcholů vstupního grafu. Druhým směrem je charakterizace omezených verzí klastrové rovinnosti a to pro klastrové grafy, kde klastry mají velikost 2 a grafem je kružnice v jednom případě a v druhém je to cesta. Charakterizaci provedeme pomocí operací redukucící klastrový graf a o těchto operacích dokážeme, že zachovávají klastrovou rovinnost. Pro tento účel jsem definoval pojem klastrového minoru, který mi umožnil studovat minimální klastrově nerovinné instance. Kromě toho uvedeme i již známe výsledky o tomto problému.
\end{document}