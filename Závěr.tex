\documentclass[12pt,a4report]{report}
\usepackage[czech]{babel}
\usepackage[T1]{fontenc}
\usepackage[utf8]{inputenc}
\usepackage{amsthm}
\usepackage{amsmath}
\usepackage{tikz}
\usepackage{float}
\usepackage{epstopdf}

\begin{document}
\author{Filip Šedivý}

\chapter{Závěr}

Nyní shrňme výsledky této práce. Ukázali jsme, že klastrovou rovinnost lze řešit v lineárním nedeterministickém čase, což nám navíc dalo i, že pokud by nás zajímala prostorová složitost, tak lze klastrovou rovinnost řešit v lineárním deterministickém prostoru.  K sestrojení algoritmu jsme využili fakt, že v případě souvislých klastrů umíme rozhodnout klastrovou rovinnost v lineárním čase. Nedeterminismus nám vydal saturátor a jen jsme museli ověřit, že všechny klastry jsou skutečně souvislé.

Dívat se na klastrovou rovinnost pomocí minimálních klastrově nerovinných minorů je směr, který se ještě nezkoušel (co je známo). Něco podobného se zkoušelo pro příbuzný problém rovinnosti částečně nakreslených grafů (viz \cite{AngeliniEtAl10} a hlavně \cite{JelinekEtAl10}). Pomocí minimálních zakázaných klastrových minorů jsme charakterizovali případy klastrových grafů s klastry velikosti 2, kde grafem je kružnice nebo cesta.

Pro další práci s charakterizací pomocí minimálních klastrově nerovinných minorů je směrem rozšířit výsledky na obecnější klastrové grafy.  Tedy kružnice či cesty s méně omezenou (nebo bez omezení) klastrovou hierarchií nebo jiná třída grafů.

\end{document}