\documentclass[12pt,a4report]{report}
\usepackage[czech]{babel}
\usepackage[utf8]{inputenc}
\usepackage{amsthm}
\usepackage{amsmath}

\newtheorem{def1}{Definice: }
\newtheorem{def2}{Definice: }
\newtheorem{def3}{Definice: }
\newtheorem{def4}{Definice:}

\begin{document}
\author{Filip Šedivý}

\chapter{Úvod}
Problém existence rovinného klastrového nakreslení grafu (dále jen klastrová rovinnost) je jedním možným zobecněním klasické grafové rovinnosti pro případ, kdy kromě vrcholů a hran máme hierarchii skupin vrcholů. Skupinu vrcholů nazýváme klastrem. Pro klastrovou rovinnost není znám polynomiální algoritmus, a není známo, zda je tento problém NP-úplný. 

\begin{def1}
Mějme graf $G=(V,H)$. Pod klastrem C budeme uvažovat podmnožinu vrcholů  $C \subseteq V$. \\
Klastrovou hierarchií jest množina klastrů, kde pro každé dva klastry $C_1$ a $C_2$ platí následující
\begin{itemize}
\item buď $C_1 \cap C_2 = \emptyset$
\item nebo $C_1 \subset C_2$
\end{itemize}
Klastrový graf je dvojice $\left(G,\mathcal C\right)$, kde G je graf a $\mathcal C$ je klastrová hierarchie.
\end{def1}

\par
Formálně se můžeme dívat na klastrovou hierarchii jako podmnožinu $\mathcal P \left({V}\right)$. To může vést k tomu, že bychom si mohli myslet, že klastrů může být velmi mnoho vzhledem k velikosti původního grafu. V kapitole složitost ukážeme, že počet klastrů je lineární vzhledem k počtu vrcholů grafu G. Kdyby takto nebyl omezen počet klastrů. Například kdyby mohl být počet klastrů superpolynomiální. Mohlo by to vyloučit, že klastrová rovinnost je z třídy P.

Klastrová rovinnost má dvě základní verze. A to nakreslená a nekreslená verze. Následující série definic je zachycuje formálněji.
\begin{def2}
Nakreslení klastru: Znázorněním klastru v rovině je topologická kružnice. Ve vnitřku kružnice leží pouze vrcholy z daného klastru a hrany grafu smí protínat hranici nakreslení klastru nejvýše jedenkrát.
Nakreslení klastrového grafu (klastrové nakreslení): Klastrový graf $\left(G,\mathcal C\right)$ je klastrově rovinný pokud graf G je rovinný a každý klastr z $\mathcal C$ jde nakreslit, tak aby se navíc žádné dvě kružnice příslušné klastrům neprotínali.
\end{def2}

\par
Omezení pro hrany v nakreslení klastru nám zaručuje, že hrany vedoucí mezi vrcholy klastru leží celé ve vnitřku nakreslení klastru. Kdyby totiž protínala kružnici, nemohla by se vrátit zpět. Podobně hrany spojující vrcholy mimo klastr musí ležet ve vnějšku. Hrana protínající nakreslenou kružnici tedy musí spojovat vrchol z klastru s vrcholem mimo klastr.

\par
Nyní máme k dispozici dostatek definic pro obě verze klastrové rovinnosti.

\begin{def3}
\begin{itemize}
\item Nenakreslená instance klastrové rovinnosti: Máme rozhodnout zda pro daný klastrový graf existuje libovolné klastrové nakreslení.
\item Nakreslená instance klastrové rovinnosti: Na vstupu máme nakreslení grafu a máme rozhodnout zda lze dokreslit klastry, tak abychom obdrželi klastrové nakreslení.
\end{itemize}
\end{def3}

Nakreslená verze klastrové rovinnosti je už na pohled omezena silnější podmínkou a to nakreslením vstupního grafu. Pokud tedy nelze dokreslit klastry tak, abychom obdrželi klastrové nakreslení, pak klastrový graf stále může být klastrově rovinný. Viz následující příklad (TODO udělat příklad)

\end{document}
