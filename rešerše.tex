\documentclass[12pt,a4report]{report}
\usepackage[czech]{babel}
\usepackage[utf8]{inputenc}
\usepackage[T1]{fontenc}
\usepackage{amsthm}
\usepackage{amsmath}
\usepackage{tikz}
\usepackage{float}

\newtheorem{theorem}{Věta}[chapter]

\theoremstyle{definition}
\newtheorem{defn}[theorem]{Definice}

\begin{document}
\author{Filip Šedivý}

\chapter{Rešerše}
V této kapitole čtenáře seznámíme se známými výsledky v oblasti klastrové rovinnosti. Jedná se o výsledky, kde pro omezenou verzi klastrové rovinnosti je znám polynomiální deterministický algoritmus pro otestování klastrové rovinnosti, případně je znám i algoritmus pro nakreslení.\\

Jedním směrem, kde omezení přineslo nějaké výsledky, je omezení se na souvislé klastry. Pro další výsledky v tomto směru se vždy požadavek na souvislost upravil. Například, že klastry indukují nejvýše dvě komponenty či některé klastry jsou obecně nesouvislé. Nyní uvedeme několik výsledků dosažených v tomto směru.

\begin{theorem}[Cortese et al. \cite{CorteseEtAl08}]
\label{souv_klastry_det_alg}
Mějme klastrový graf $(G, \mathcal C)$, kde každý klastr $K \in \mathcal C$ je souvislý. Pak existuje lineární deterministický algoritmus rozhodující zda $(G, \mathcal C)$ je klastrově rovinný.
\end{theorem}

Tento výsledek využijeme později při konstrukci lineárního nedeterministického algoritmu pro obecné klastrové grafy (viz kapitola \ref{slozitost}).

Před uvedením dalšího výsledku uvedeme definici takzvaného úplně souvislého klastrové grafu.

\begin{defn}
Klastrový graf $(G, \mathcal C)$ je \textit{úplně souvislý}, pokud pro každý klastr $K \in \mathcal C$ je $K$ souvislý a i $V \setminus K$ je souvislý.
\end{defn}

\begin{theorem}[Cornelsen a Wagner \cite{CornelsenWagner03}]
Úplně souvislý klastrový graf $(G, \mathcal C)$ je  klastrově rovinný $\iff$ $G$ je rovinný.
\end{theorem}

Rovinnost lze rozpoznávat v lineárním čase. A navíc je i možné získat v tomto případě v lineárním čase klastrové nakreslení.

\begin{theorem}[Jelínek et al. \cite{JelinekEtAl08}]
Mějme nakreslený klastrový graf $(G, \mathcal C, \rho)$. Pokud každý klastr $K \in \mathcal C$ indukuje nejvýše dvě komponenty, pak existuje lineární algoritmus pro rozhodnutí, zda $(G, \mathcal C)$ je klastrově rovinný.
\end{theorem}

\begin{theorem}[Gutwenger et al. \cite{GutwengerEtAl02}]
Pokud všechny nesouvislé klastry klastrového grafu $(G, \mathcal C)$ leží na stejné cestě začínající v kořeni klastrové hierarchie, pak pro $(G, \mathcal C)$ lze v kvadratickém čase rozhodnout, zda je klastrově rovinný.
\end{theorem}

\begin{theorem}[Goodrich et al. \cite{GoodrichEtAl05}]
Mějme klastrový graf $(G, \mathcal C)$. Nechť pro každý nesouvislý klastr $K \in \mathcal C$ platí, že jeho rodič a sourozenci v klastrové hierarchii jsou souvislé klastry. Potom pro $(G, \mathcal C)$ lze v kvadratickém čase rozhodnout, zda je klastrově rovinný.
\end{theorem}

\begin{theorem}[Gutwenger et al. \cite{GutwengerEtAl02}]
Mějme klastrový graf $(G, \mathcal C)$. Každý nesouvislý klastr $K$ má souvislého rodiče a souvislé komponenty $K$ mají napojení mimo rodiče. Pak je algoritmus pracující v polynomiálním čase rozhodující o klastrové rovinnosti a dávající klastrové nakreslení v případě kladné odpovědi.
\end{theorem}

Další směr, který přinesl výsledky, se týká takzvaných placatých klastrových grafů. Jedná se o omezení klastrové hierarchie, kde klastry jsou po dvou disjunktní (nepočítaje klastr obsahující všechny vrcholy a jednovrcholové klastry).

\begin{defn}
Klastrový graf $(G, \mathcal C)$ je \textit{placatý} pokud všechny klastry kromě kořene (klastru obsahující všechny vrcholy) mají jako rodiče kořen (jednovrcholové klastry ignorujeme).
\end{defn}

\begin{theorem}[Jelínková et al. \cite{JelinkovaEtAl07}]
Mějme placatý klastrový graf  $(G, \mathcal C)$, kde $G$ je kružnice. Pokud každý klastr obsahuje nejvýše tři vrcholy, pak lze v polynomiálním čase rozhodnout, zda je  $(G, \mathcal C)$ klastrově rovinný.
\end{theorem}

\begin{theorem}[Cortese et al \cite{CorteseEtAl04}]
Mějme placatý klastrový graf  $(G, \mathcal C)$, kde $G$ je kružnice. Pokud klastry jsou uspořádané do cyklu nebo cesty, pak lze v polynomiálním čase rozhodnout, zda je  $(G, \mathcal C)$ klastrově rovinný.
\end{theorem}

\begin{theorem}[Cortese et al \cite{CorteseEtAl09}]
Mějme placatý klastrový graf  $(G, \mathcal C)$, kde $G$ je kružnice. Pokud klastry jsou uspořádané do nakresleného rovinného grafu, pak lze v polynomiálním čase rozhodnout, zda je  $(G, \mathcal C)$ klastrově rovinný.
\end{theorem}

\begin{theorem}[Jelínková et al. \cite{JelinkovaEtAl07}]
Mějme placatý klastrový graf  $(G, \mathcal C)$, kde $G$ je 3-souvislý a všechny stěny mají velikost nejvýše 4, pak lze v polynomiálním čase rozhodnout, zda je  $(G, \mathcal C)$ klastrově rovinný.
\end{theorem}

\begin{theorem}[DiBattista a Frati \cite{DiBattistaFrati07}]
Mějme placatý nakreslený klastrový graf  $(G, \mathcal C, \rho)$, kde všechny stěny mají velikost nejvýše 5 a mějme pevné nakreslení grafu $G$, pak lze v polynomiálním čase rozhodnout, zda je  $(G, \mathcal C)$ klastrově rovinný.
\end{theorem}

Jiným směrem bylo omezení grafu na vrcholově 3 souvislý.

\begin{theorem}[Jelínková et al. \cite{JelinkovaEtAl07}]
Mějme klastrový graf $(G, \mathcal C)$, kde klastry mají velikost nejvýše 3 (kromě kořene) a $G$ je vrcholově 3-souvislý, pak lze v polynomiálním čase rozhodnout, zda je  $(G, \mathcal C)$ klastrově rovinný. 
\end{theorem}

\begin{theorem}[Jelínková et al. \cite{JelinkovaEtAl07}]
Mějme klastrový graf $(G, \mathcal C)$, kde klastry mají velikost nejvýše 3 (kromě kořene) a $G$ je dělení vrcholově 3-souvislého multigrafu, který má $\mathcal O(1)$ vrcholů, a G má všechny stupně sudé, pak lze v polynomiálním čase rozhodnout, zda je  $(G, \mathcal C)$ klastrově rovinný. 
\end{theorem}

Ještě jedním směrem je omezení počtu klastrů. Konkrétně na dva klastry, viz články \cite{BiedlI98}, \cite{BiedlII98}, \cite{BiedlIII98}. Dokazuje se v nich (mimo jiné) lineární algoritmus pro situaci, kdy množina vrcholů je rozdělena na právě dva neprázdné disjunktní klastry (nepoužívá se v nich terminologie klastrové rovinnosti, ale je to ekvivalentní). 

Závěrem je nutno poznamenat, že jinak se jedná o otevřené problémy. Například pro obecné placaté grafy se neví, jestli lze efektivně rozhodovat o klastrové rovinnosti. Pro obecnou klastrovou rovinnost je otevřeným problémem zda patří do P nebo jestli je NP-úplný.

% Dál nekopírovat

\begin{thebibliography}{9}

\bibitem{CorteseEtAl08}
	Cortese, P.F., Di Battista, G., Frati, F., Patrignani, M., Pizzonia, M.: C-planarity of cconnected clustered graphs. J. Graph Alg. Appl. 12(2), 225–262 (2008)

\bibitem{CornelsenWagner03}
	S. Cornelsen and D. Wagner. Completely connected clustered graphs. Journal of Discrete Algorithms, 4(2):313–323, 2006.

\bibitem{JelinekEtAl08}
	V. Jelínek, E. Jelínková, J. Kratochvíl, B. Lidický: Clustered Planarity: Embedded Clustered Graphs with Two-Component Clusters (extended abstract), Proceedings of Graph Drawing 2008, LNCS 5417 (2009), 121-132

\bibitem{GutwengerEtAl02}
	Gutwenger, C., Jünger, M., Leipert, S., Mutzel, P., Percan, M., Weiskircher, R.: Advances in c-planarity testing of clustered graphs. In: Goodrich, M.T., Kobourov, S.G. (eds.) GD’02. LNCS, vol. 2528, pp. 220–235. Springer (2002)

\bibitem{GoodrichEtAl05}
	Goodrich, M.T., Lueker, G.S., Sun, J.Z.: C-planarity of extrovert clustered graphs. In: Healy,P., Nikolov, N.S. (eds.) GD’05. LNCS, vol. 3843, pp. 211–222. Springer (2006)

\bibitem{JelinkovaEtAl07}
	Jelínková, E., Kára, J., Kratochvíl, J., Pergel, M., Suchý, O., Vyskocil, T.: Clustered planarity: Small clusters in cycles and eulerian graphs. J. Graph Alg. Appl. 13(3), 379–422 (2009)

\bibitem{CorteseEtAl04}P. F. Cortese, G. Di Battista, M. Patrignani, and M. Pizzonia. Clustering cycles into cycles of clusters. Journal of Graph Algorithms and Applications,
9(3):391–413, 2005.

\bibitem{CorteseEtAl09}
	P. F. Cortese, G. Di Battista, M. Patrignani, and M. Pizzonia. On embedding a cycle in a plane graph. In Proceedings of 13th International Symposium on Graph Drawing 2005, volume 3843 of LNCS, pages 49–60. Springer, Heidelberg, 2006.

\bibitem{DiBattistaFrati07}
	Di Battista, G., Frati, F.: Efficient c-planarity testing for embedded flat clustered graphs with small faces. In: Hong, S.H., Nishizeki, T., Quan, W. (eds.) GD’07. LNCS, vol. 4875, pp. 291–302. Springer (2008)

\bibitem{BiedlI98}
	Biedl, T.: Drawing planar partitions I; LL-drawings and LH-drawings. Technical Report RRR 11-98, RUTCOR, Rutgers University, 1998.

\bibitem{BiedlII98}
	 Bied, T., Kaufmannl, M., Mutzel, P.: Drawing planar partitions II: 
HH-drawings. Technical Report RRR 12-98, RUTCOR, Rutgers University, 1998.

\bibitem{BiedlIII98}
	Biedl, T.: Drawing planar partitions I; LL-drawings and LH-drawings. Technical Report RRR 11-98, RUTCOR, Rutgers University, 1998.

\end{thebibliography}


\end{document}